\documentclass{article}

\usepackage[utf8]{inputenc}
\usepackage{listings}
\usepackage[most]{tcolorbox}
\usepackage{enumitem}

\newtcblisting{messageshell}{colback=black,colupper=white,colframe=yellow!75!black,
listing only,listing options={language={}, basicstyle=\small\ttfamily, breaklines=true,aboveskip=0pt, belowskip=0pt},
every listing line={}}

\title{Stochastic Simulation Lab \#2 report}
\author{Shitong CHAI}
\date{}

\begin{document}
\maketitle 

\begin{enumerate}
\item{\textbf{Find “Matsumoto Home page” and the last C implementation of the original Mersenne Twister}}

I have found the source file and included the header file in my source code as shown below.
\begin{lstlisting}[language=C]
#include<stdio.h>
#include<stdlib.h>
#include<math.h>
#include "mt19937ar.h"
\end{lstlisting}

\item{\textbf{Generation of uniform random numbers between A and B}}

Using the function "genrand\_real2" defined in "mt19937ar.h", and the formula $y=(b-a)x+a$, the function "uniform" is written as follows.

\begin{lstlisting}[language=C]
double uniform(double a, double b)
{
/*
* :argument 
*       a,b are the endpoints of an interval [a,b]. 
* :return  
*       A random number in the interval [a,b].
* :usage
*       To generate a uniformly distributed random number.
*/

  return (b - a) * genrand_real2() + a;
}
\end{lstlisting}
The numbers generated:
\begin{messageshell}
1000 outputs of uniform(2, 4)
2.49713780 2.44514696 2.22225526 3.91257279 3.96927063
3.55732628 3.56256356 2.10599778 2.37727890 3.20666351
3.19221428 3.51001672 2.30116489 2.68895420 3.97695208
2.53242437 3.79916178 3.49990524 2.82590178 2.53024722
2.66212271 2.59135096 3.33176057 2.82164228 3.80438722
3.84649598 2.03378356 3.11013135 3.21082118 3.50757487
3.30054610 2.33775862 2.28211709 2.12631479 2.33301166
3.53575071 2.61605384 3.96955125 2.64963992 3.73956981
2.97869328 3.73705059 3.37390305 2.77389266 3.34835932
3.10385543 2.46756682 3.04638593 2.98312384 3.50750636
3.90102717 2.10234538 3.14275137 3.72740918 2.38910947
3.12315169 3.27367910 2.35223606 3.01291095 2.87258683
3.28640525 3.93799280 3.56047723 2.08434517 3.52266207
2.36136704 3.03291216 3.39779681 2.86930854 2.99354783
2.01623423 2.55503779 3.44462630 3.35810942 2.47143677
3.01739136 3.98250611 3.35018357 3.77793906 3.41130608
2.59705303 2.19557944 3.08332442 3.27658980 2.59512475
3.74351950 2.12655604 3.22298022 3.19235552 3.42305051
3.97138990 2.57201724 3.67432210 2.50194329 3.11569273
2.86237084 3.40335574 2.93130950 3.40019178 2.48884262

\end{messageshell}
\item{\textbf{Reproduction of discrete empirical distributions}}
\begin{enumerate}[label=(\alph*)]
\item{Implement a function to reproduce the histogram given below and cumulate.}

The function to reproduce the histogram $P\{X=A\}=0.5,P\{X=B\}=0.1,P\{X=C\}=0.4$ is given below.

\begin{lstlisting}[language=C]
int species_distribution(void)
{
/*
* :return  
*       A random number which corresponds to a species.
* :usage
*       To generate a species randomly with a given C.D.F.
*/
  double    Array[3] = {0.5, 0.6, 1};
  double    rand_num = genrand_real2();
  for(int i=0; i<3; i++)
    {
        if(rand_num < Array[i])
        {
            return i;
        }
    }
  return -1;
}
\end{lstlisting}

The distribution collected with the function defined above is as follows.

\begin{messageshell}

 Probabilistic distribution percentage:
 49.10%  8.30%  42.60%

 Accumulative distribution(repartition) percentage:
 49.10%  57.40%  100.00%

 1000 000 drawings:

 Probabilistic distribution percentage:
 49.98%  10.05%  39.97%

 Accumulative distribution(repartition) percentage:
 49.98%  60.03%  100.00%
\end{messageshell}

\item{Implement another function, test this function with your own data and check a simulated distribution.}

The other function is defined below.

\begin{lstlisting}[language=C]
typedef int (*fun_ptr)();
void test_reproduction(int drawings, int size, fun_ptr distribution)
{
/*
* :argument 
*       drawings is the number of samples, size is the size of the
*       given discrete repartition function array, distribution is
*       the distribution from which random classes are generated. 
* :usage
*       To collect the statistical distribution of the generated
*       random classes and print them to the screen.
*/
    int *Array = malloc(size * sizeof(int));
    for(int i=0; i < size; i++)
    {
        Array[i] = 0;
    }
    for(int i=0; i < drawings; i++)
    {
        int rand_species = distribution();
        for(int j=0; j < size; j++)
            if(rand_species == j)
                Array[j]++;
    }
    printf("\n Probabilistic distribution percentage:\n");
    for(int i=0; i < size; i++)
    {
        printf(" %.2lf%% ", 100 * Array[i]/(double)drawings);
    }

    printf("\n");
    printf("\n Accumulative distribution(repartition) percentage:\n");
    for(int i=1; i < size; i++)
    {
        Array[i] += Array[i - 1];
    }
    for(int i=0; i < size; i++)
    {
        printf(" %.2lf%% ", 100 * Array[i]/(double)drawings);
    }
    printf("\n");
}
\end{lstlisting}

I use this simulated distribution to generate my own data:

\begin{lstlisting}[language=C]
int another_distribution(void)
{
/*
* :return  
*       A random number which corresponds to a class.
* :usage
*       To generate a class randomly with a given repartition 
*       function which is a simulated distribution.
*/
    double      Array[5] = {0.2, 0.3, 0.5, 0.8, 1};
    double      rand_num = genrand_real2();
    for(int i=0; i<5; i++)
    {
        if(rand_num < Array[i])
        {
            return i;
        }
    }
    return -1;
}
\end{lstlisting}
My own data generated using this distribution is checked with function "test\_reproduction", result is given below.
\begin{messageshell}
 1000 drawings:

 Probabilistic distribution percentage:
 19.20%  9.10%  19.90%  31.00%  20.80%

 Accumulative distribution(repartition) percentage:
 19.20%  28.30%  48.20%  79.20%  100.00%

 1000 000 drawings:

 Probabilistic distribution percentage:
 20.10%  10.00%  19.97%  29.92%  20.01%

 Accumulative distribution(repartition) percentage:
 20.10%  30.11%  50.07%  79.99%  100.00%

\end{messageshell}
\end{enumerate}
\item{\textbf{Reproduction of continuous distributions.}}

With the formula $y=-M\ln(1-x)$, we can generate numbers according to a exponential distribution. This function is written as follows.

\begin{lstlisting}[language=C]
double negative_exponential(double M_coeff)
{
/*
* :argument 
*       M_coeff is the coefficient M in the formula of negative 
*       exponential distribution function. 
* :return  
*       A random number following the PDF: f(x)=exp(-x/M)/M.
* :usage
*       To generate a negative exponentially distributed 
*       random number.
*/
    double      rand_num = genrand_real2();
    return -M_coeff*log(1-rand_num);
}
\end{lstlisting}

The cumulated result is as follows.
\begin{messageshell}

1000 drawings(P.D.F.):
10.10% 8.30% 7.00% 6.70% 5.30% 6.90% 5.80% 5.90% 3.50% 4.20% 3.10% 3.70% 2.40% 2.20% 2.00% 1.90% 2.00% 1.00% 2.00% 1.00%

1000000 drawings(P.D.F.):
9.51% 8.62% 7.82% 7.05% 6.33% 5.76% 5.23% 4.71% 4.28% 3.88% 3.50% 3.15% 2.87% 2.59% 2.33% 2.14% 1.90% 1.74% 1.57% 1.44% 
\end{messageshell}
\item{\textbf{Simulating non reversible distribution laws.}}
With the Box and Muller function: $x_1=\cos(2\pi Rn_2)\sqrt{-2\ln(Rn_1)}$, we can generate numbers following distribution $N(0,1)$. The function to do this is as follows.
\begin{lstlisting}[language=C]
double norm_box_muller(void)
{
/*
* :return  
*       A random number following standard normal distribution.
* :usage
*       To generate a random number following the distribution N(0,1).
*/
    double rn1 = genrand_real2();
    double rn2 = genrand_real2();
    return cos(2 * M_PI * rn2) * sqrt(-2*log(rn1));
}  
\end{lstlisting}

The collected data ($N(0,1)$ and $N(10,3^2)$) which is shown below fits the known statistics for the Gaussian distribution.

\begin{messageshell}

1000 drawings:
0.02 0.15 0.31 0.34 0.15 0.02

1000000 drawings:
0.02 0.14 0.34 0.34 0.14 0.02

Gaussian distribution with an average of 10 and with a standard deviation at 3
1000 drawings:
0.00 0.00 0.00 0.00 0.00 0.01
\end{messageshell}
\item{\textbf{Find libraries in C/C++ and Java that generate random variates like you did in the previous questions.}}
GNU Scientific Library has many RNGs covering almost all of the RNGs in the previous questions.

"mt19937ar.h" can be replaced by including "gsl\_rng.h", the mt19937 RNG algorithm is declared as follows.
\begin{lstlisting}[language=C]
GSL_VAR const gsl_rng_type *gsl_rng_mt19937;
\end{lstlisting}

Uniform distribution can be obtained by this function:
\begin{lstlisting}[language=C]
double gsl_ran_flat(const gsl_rng * r, double a, double b);
\end{lstlisting}

These functions can be used to generated a random number following a discrete random distribution:
\begin{lstlisting}[language=C]
gsl_ran_discrete_t * gsl_ran_discrete_preproc(size_t K, const double * P);
size_t gsl_ran_discrete(const gsl_rng * r, const gsl_ran_discrete_t * g);
\end{lstlisting}

The gaussian distributed random number can be generated using this function:
\begin{lstlisting}[language=C]
double gsl_ran_gaussian(const gsl_rng * r, double sigma);
\end{lstlisting}
\end{enumerate}
\end{document}
