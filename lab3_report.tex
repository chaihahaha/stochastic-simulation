\documentclass{article}


\usepackage[utf8]{inputenc}
\usepackage{listings}
\usepackage[most]{tcolorbox}
\usepackage{enumitem}

\lstset{
  breaklines=true
}

\newtcblisting{messageshell}{colback=black,colupper=white,colframe=yellow!75!black,
listing only,listing options={language={}, basicstyle=\small\ttfamily, breaklines=true,aboveskip=0pt, belowskip=0pt},
every listing line={}}

\title{Stochastic Simulation Lab \#3 report}
\author{Shitong CHAI}
\date{}

\begin{document}
\maketitle 

\begin{enumerate}
\item{\textbf{Compute $\pi$ with the Monte Carlo method }}

The function "compute\_pi" defined below used the formula $\pi = 4 \frac{S_\text{quarter circle}}{S_\text{square}} = 4 \frac{N_\text{points inside}}{N_\text{total}}$ to compute $\pi$:
\begin{lstlisting}[language=C]
double compute_pi(unsigned long long points)
{
    
    /*
     * :argument
     *     points: the number of points
     *             generated to get Pi.
     * :return
     *  The value of Pi computed with Monte Carlo 
     *     method.
     * :usage
     *     Call the function with drawings, the bigger
     *     drawings is, the more precise the answer 
     *     will be.
     */
    double x, y;
    unsigned long long    counter=0;
    for(unsigned long long i=0; i < points; i++)
    {
        x = genrand_real2();
        y = genrand_real2();
        if((x * x + y * y) < 1)
            counter++;
    }
    return 4 * (double)counter/(double)points;

}
\end{lstlisting}
The experiment result is as follows:
\begin{messageshell}
1) Compute Pi with the Monte Carlo method:
1000 points:
Pi ~ 3.172000
1000 000 points:
Pi ~ 3.139400
1000 000 000 points:
Pi ~ 3.141541
\end{messageshell}
The converging process from 100 000 drawings to 60 000 100 000 drawings is as follows:
\begin{messageshell}
drawings: 100000, pi ~ 3.140760
drawings: 10000100000, pi ~ 3.141608
drawings: 20000100000, pi ~ 3.141617
drawings: 30000100000, pi ~ 3.141613
drawings: 40000100000, pi ~ 3.141609
drawings: 50000100000, pi ~ 3.141604
drawings: 60000100000, pi ~ 3.141599
\end{messageshell}
So to get a precision at $10^{-2}$, I will need at least 100 000 drawings. To get a precision at $10^{-4}$, I will need at least 60 000 100 000 drawings. This corresponds to the square root converging speed.
\item{\textbf{Propose a loop on this function to compute n independent experiments }}

The function defined below is used to get n independent experiment results:
\begin{lstlisting}[language=C]
double *independent_experiments(unsigned long long n, unsigned long long points)
{
    /*
     * :argument 
     *     n: the number of independent experiments
     *     points: the number of points generated.
     * :return
     *     array storing n experiment results
     * :usage
     *     Run MC simulation n times with fixed
     *     points generated.
     */
    double *results = malloc(n * sizeof(double));
    printf("Running %llu independent experiments\n", n);
    for(unsigned long long i=0; i < n; i++)
    {
	results[i] = compute_pi(points);
    }
    return results;
}
\end{lstlisting}
The experiment results is as follows:
\begin{messageshell}
2) n independent experiments
Running 10 independent experiments
Experiment 0: 3.139964
Experiment 1: 3.143592
Experiment 2: 3.140024
Experiment 3: 3.139328
Experiment 4: 3.138552
Experiment 5: 3.142028
Experiment 6: 3.143480
Experiment 7: 3.139464
Experiment 8: 3.141672
Experiment 9: 3.140424
Final result(average): 3.140853
\end{messageshell}
\item{\textbf{Computing of confidence intervals around the mean}}

The function "sigma2" is used to compute estimated variance, the function "tn\_1\_distribution" is used to simulate the distribution $t_{n-1,1-\frac{\alpha}{2}}$, and the function "confidence\_radius" is used to compute the confidence radius. Code is as follows:
\begin{lstlisting}[language=C]
double sigma2(double *a, double mean, unsigned long long n)
{
    /*
     * :argument
     *     n: the number of independent experiments 
     *     mean: the average value of array a
     *     a: array storing n experiment results 
     * :return 
     *     estimate without bias of sigma^2 variance
     * :usage
     *     compute the estimated variance of array a 
     */
     double summary=0;
     for(unsigned long long i=0; i < n; i++)
     {
         summary += (a[i]-mean) * (a[i]-mean);
     }
     return summary/(n-1);
}
double tn_1_distribution(unsigned long long n)
{
    if(n<=0)
        return -1;
    double t_n[35]={-1, 12.706, 4.303, 3.182, 2.776, 
    2.571, 2.447, 2.365, 2.308, 2.262, 2.228, 2.201, 
    2.179, 2.160, 2.145, 2.131, 2.120, 2.110, 2.101, 
    2.093, 2.086, 2.080, 2.074, 2.069, 2.064, 2.060, 
    2.056, 2.052, 2.048, 2.045, 2.042, 2.021, 2.000, 
    1.980, 1.960}; 
    if(n<=30)
        return t_n[n];
    else if(n<=40)
        return t_n[31];
    else if(n<=80)
        return t_n[32];
    else if(n<=120)
        return t_n[33];
    else
        return t_n[34];
}
double confidence_radius(double *a, double mean, unsigned long long n)
{
    /*
     * :argument
     *     n: the number of independent experiments 
     *     mean: the average value of array a
     *     a: array storing n experiment results 
     * :return 
     *     confidence radius
     * :usage
     *     compute the confidence radius of n 
     *     independent experiments 
     */
    
    return tn_1_distribution(n)*sqrt(sigma2(a, mean, n)/n);
}
\end{lstlisting}
And the experiment result is as follows:
\begin{messageshell}
3) Computing of confidence intervals around the mean
Running 10 independent experiments
Mean value: 3.142052
Confidence interval: [3.140781, 3.143322]
Confidence level: 95%
\end{messageshell}
\end{enumerate}
\end{document}
